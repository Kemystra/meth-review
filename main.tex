\documentclass[a4paper, 12pt]{article}
\usepackage[american]{babel}
\usepackage{csquotes}
\usepackage[style=apa,backend=biber,natbib=true]{biblatex}
\DeclareLanguageMapping{american}{american-apa}
\usepackage[T1]{fontenc}
\usepackage[utf8]{inputenc}
\usepackage{mathptmx}
\usepackage{enumerate}
\usepackage{amsmath} 
\usepackage[margin=0.5in]{geometry}

\addbibresource{main.bib}


\renewcommand{\baselinestretch}{1.0}

\newcommand\nd{\textsuperscript{nd}\xspace}
\newcommand\rd{\textsuperscript{rd}\xspace}
\newcommand\nth{\textsuperscript{th}\xspace} %\th is taken already

\setlength\parindent{0pt} % set paragraph indent to zero

% fill up your name, ID and paper title here
\author{
IZZMINHAL AKMAL BIN NORHISYAM \quad 242UC240JF \quad contribution1 \\
CHOW YING TONG \quad 242UC244NK \quad contribution2\\
CHOONG JIA XUEN \quad 242UC244K1 \quad contribution3\\
Student Name4 \quad Student ID4 \quad contribution4\\
}
\title{ REVIEW REPORT  Title  }



\begin{document}
\maketitle


\section{Introduction and Problem Statement}
Write the introduction here. What is the background of the problem? Why is there a problem? What are the previous work/research to attempt the problem? 

Problem statement in a new paragraph, follow by objectives.

Citation example: \cite{Termenchy15} and \cite{Jones19}.

\section{Literature Review}
\subsection{Holistically Analyzing the Carbon Footprint of AI}
The work by \citet{Wu2022} provides a foundational, holistic analysis of the sources of AI's greenhouse gas emissions, shifting the focus beyond prior research that was often limited to the carbon footprint of training a single, massive model. The authors contend that this narrow, model-centric view has become insufficient in the face of the "super-linear growth" of the entire AI ecosystem, a trend characterized by exponential increases in model complexity, vast volumes of training data, and the ever-expanding hardware infrastructure required to support them (p. 2). To provide a more accurate assessment, their analysis expands the scope to encompass two critical dimensions: the end-to-end machine learning pipeline (from data acquisition to inference) and the life cycle of the hardware itself. Critically, the authors introduce a practical framework that distinguishes between operational carbon, defined as the emissions from electricity consumed during use, and embodied carbon, which are the often-hidden emissions generated from the entire manufacturing supply chain (p. 2). Other than that, the authors provide several solutions that could be employed to mitigate the carbon footprint of AI.

\subsection{Holistically Analyzing the Ecological Costs of AI}
The analysis by \citet{Zhuk2023} offers a comprehensive interdisciplinary approach to understanding the hidden ecological costs associated with the development and deployment of AI technologies. Unlike previous studies that often focus narrowly on energy consumption during AI model training, Zhuk expands the scope to include not only the energy-intensiveness of computing operations but also the broader environmental impacts such as the lifecycle emissions of hardware manufacturing, data center energy demands, e-waste generation from rapid equipment obsolescence, and the ecological disruptions caused by AI infrastructure expansion. Importantly, the work highlights how non-renewable energy reliance leads to increased carbon emissions and poses systemic obstacles to sustainable ecological development. Moreover, the study integrates ethical and political-legal dimensions, underscoring how AI-related errors and algorithmic biases can exacerbate environmental injustice and inequality. The author calls for holistic strategies aimed at energy-efficient algorithm design, renewable energy adoption, responsible e-waste management, and the formulation of binding ethical-legal frameworks at multiple governance levels to harmonize AI progress with environmental sustainability.

\section{Research Methodology}
\subsection{Data Collection and Sourcing}
Based on the study conducted by \citet{Wu2022}, the data was collected from multiple sources, from internal production systems at Meta, external public reports, and experimental modeling to create a holistic view of AI's environmental footprint. The primary data source consisted of extensive internal infrastructure telemetry from Meta's large-scale AI systems, encompassing operational metrics such as data center power consumption logs, GPU utilization rates, model training job durations, and data ingestion bandwidth statistics. To account for the manufacturing footprint, the authors sourced Life Cycle Inventory (LCI) data from external databases and manufacturer publications, such as Apple's Product Environmental Reports, which served as a proxy for their own hardware components (p. 5). For the purpose of comparative analysis, the study also incorporated the publicly available carbon footprint results of other prominent models, including GPT-3 and Meena, to contextualize their own findings (p. 5). Finally, to model the energy consumption of specific edge-computing use cases like federated learning, the authors constructed their estimations using a combination of experimental measurements, including device power profiles from Android (p. 19) and network bandwidth data from public sources.

In addition, the study by \citet{Zhuk2023} leverages comprehensive data on the environmental impacts of AI that extends to manufacturing supply chains, hardware lifecycle assessments, and ecosystem-level influences. The data collection process integrates metrics of energy consumption and associated carbon emissions at each stage of AI system deployment, including electronic waste generation and renewable energy utilization. Ethical-legal impact data were also considered by incorporating existing regulatory frameworks and case studies on AI-related environmental injuries and risks to communities, thus providing a multidisciplinary dataset that enriches the environmental footprint assessment of AI technologies.

\subsection{Core Methodology and Algorithms} 
\citet{Wu2022} uses a Carbon Footprint Analysis framework that is designed to holistically estimate the carbon footprint of AI, taking into account the complete machine learning (ML) pipeline end-to-end: data collection, model exploration and experimentation, model training, model optimization and run-time inference, while also taking into account the emissions across the life cycle of hardware systems, from manufacturing to operational use (p. 2). The calculation mechanism is divided into two parts: operational carbon and embodied carbon. Operational carbon is calculated by measuring the total energy consumption, location-based carbon intensity for electricity grids, while using a Power Usage Effectiveness (PUE) factor of 1.1 (p. 5). On the other hand, embodied carbon (carbon footprint of AI hardware) is quantified using LCA (Life Cycle Analysis). GPU-based AI training systems are assumed to have a similar embodied footprint as Apple's 28-core CPU with dual AMD Radeon GPUs in production, while for CPU-only AI training systems are assumed to have half that amount.  Based on the characterization of model training and inference at Meta, an average utilization rate of 30\% to 60\% over the 3-5 year lifetime for servers (p. 5). Combining all metrics above, the embodied carbon footprint is estimated to be: 
\begin{equation}
	CO_2^{\text{embodied}} = \sum_i \frac{\text{Time}}{\text{Lifetime}} CO_2^{\text{embodied}}(AI_{\text{System}})(i)
\end{equation}

Building upon Wu et al.'s (2022) carbon footprint analysis framework that distinguishes between operational and embodied carbon, \citet{Zhuk2023}'s study further expands the methodology by incorporating ethical-legal considerations and non-energy ecological costs. The framework integrates a lifecycle assessment (LCA) method for quantifying embodied carbon in AI hardware, accounting for the full manufacturing-to-disposal chain, and includes additional environmental externalities such as e-waste accumulation and resource depletion. The operational carbon footprint is computed with energy consumption data adjusted for grid carbon intensity and power usage effectiveness (PUE). The methodology incorporates a multidisciplinary approach by combining quantitative carbon accounting with qualitative assessments of environmental justice, policy gaps, and ethical implications of AI deployment, thereby enabling a holistic evaluation of AI's environmental and societal impacts.

\section{Result}
\citet{Wu2022} presents an important finding: the authors' estimations reveal that when considering the entire life cycle, the split between the operational carbon and embodied carbon is roughly 70\% / 30\% respectively, for large-scale machine learning tasks (p. 5). This result is significant as it demonstrates that the manufacturing supply chain (embodied carbon) is a major contributor to AI's total environmental impact, a factor often overlooked in prior research that focused solely on electricity consumption. They also note that, after considering carbon-free energy sources such as solar, the operational carbon footprint can be reduced significantly, leaving the manufacturing carbon cost as the dominating source of AI's carbon footprint (p. 5). Furthermore, the results reveal that the source of operational carbon emissions varies significantly depending on the AI task. For a large-scale language model (LM), the inference phase was found to dominate the carbon footprint (65\%) In contrast, for deep learning recommendation models (RM1–RM5), the carbon footprint was split more evenly between the training and inference phases (pp. 4–5). This finding disproves the common assumption that model training is always the most carbon-intensive part of an AI model's life cycle.

\citet{Zhuk2023}'s analysis shows a range of environmental, ethical and politicallegal issues associated with the training, use and development of artificial intelligence, which consumes a significant amount of energy (mainly from non-renewable sources). This leads to an increase in carbon emissions and creates obstacles to further sustainable ecological development. Improper disposal of artificial intelligence equipment exacerbates the problem of e-waste and pollution of the planet, further damaging the environment. Errors in artificial intelligence algorithms and decision-making processes
lead to environmental injustice and inequality. AI technologies may disrupt
natural ecosystems, jeopardizing wildlife habitats and migration patterns.

\section{Discussions}
\subsection{Advantages of Research Methodologies Used}
The first paper that was covered is that of \citet{Wu2022}. The main advantage of their research is the use of real-world production data from Meta, one of the largest AI infrastructures in the world. This is crucial because it gives the findings credibility that would otherwise be unattainable using simulated data, or in a small-scale academic setting. It paints an accurate picture of what AI's carbon footprint is like in real-world environments, thus eliminating the need to cross-check their findings with other institutions. Moreover, the framework with which the carbon footprint is calculated is actionable, pragmatic and easy to understand. As opposed to a complete LCA, which is often overly complex and difficult to digest, the distinction of operational carbon and embodied carbon is straightforward and easy for non-technical people such as AI engineers and managers to comprehend.

The methodology used in \citet{Zhuk2023}'s research article offers several notable advantages. Its interdisciplinary approach holistically covers not only AI's direct operational carbon footprint but also incorporates lifecycle environmental costs such as hardware manufacturing, e-waste, and ecosystem impacts. Integrating ethical and legal perspectives broadens the scope to include societal and governance dimensions, which is crucial for framing sustainable AI development in a just and responsible manner. This comprehensive and systemic view transcends the limitations of narrowly quantitative carbon footprint studies, enabling a more nuanced understanding of AI's true environmental costs and policy implications.

\subsection{Limitations of Research Methodologies Used}
Although the research conducted by \citet{Wu2022} is rigorous and remarkable, it lacks generalizability across different AI ecosystems. The research findings are largely derived from Meta, a multinational conglomerate with one of the most complete AI infrastructures in the world, and has massive investments and commitments to renewable energy. Therefore, the findings might not be applicable to other institutions and settings, such as smaller companies that do not have access to the same resources as Meta. Besides that, the calculation of the embodied carbon footprint is an estimation, not the actual value. Due to the lack of publicly available data from hardware manufacturers, the authors had to rely on proxy data from other companies for certain hardware components. For example, the research assumes that a GPU-based AI training system has the same carbon footprint as Apple's 28-core CPU with dual AMD Radeon GPUs (p. 5). As a result, the actual embodied carbon footprint could be higher or lower than the estimated value. Furthermore, the research framework does not take into account the end-of-life phase for hardware components, which results in the oversight of the carbon emissions produced when hardware is incinerated or left to decompose in landfills.

Despite its strengths, the methodology in \citet{Zhuk2023}'s research article lacks formal quantitative modeling or formulaic analysis constrains the precision and reproducibility of environmental impact estimates, making it difficult to benchmark findings against other lifecycle assessments. The research primarily relies on secondary data and existing literature, which may limit measurement accuracy and fail to capture emerging empirical evidence from real-time AI deployments. Additionally, the complexity of integrating ethical-legal issues poses challenges for producing clear actionable outcomes and practical guidelines for AI developers and policymakers.

\subsection{Contribution of the Research}
\citet{Wu2022} presents a practical and industry-tested framework for quantifying AI's carbon footprint. Although the concept of operational and embodied carbon had been introduced prior, the paper puts a credible, industrial-scale number on it, estimating the embodied carbon accounts for at least 30\% of the total emissions (p. 5). It lays the foundation for more discussion and research on the subject of reducing the carbon emissions of hardware components. Secondly, beyond quantifying the problem, a significant contribution of the research is its demonstration of a practical, industry-tested strategy for mitigating AI's carbon footprint. The authors advocate for an approach of continuous hardware-software co-design, which they present not as a single solution but as an iterative, full-stack optimization process (p. 5). The efficacy of this strategy is illustrated through a case study where the operational carbon footprint of a language model was reduced by over 800 times (p. 6). This was not a single optimization but a series of compounding gains across different layers. At the software and platform layers, optimizations included application-level caching (a 6.7x efficiency improvement) and algorithmic techniques like numerical optimization to 16-bit precision (a 2.4x gain) and the use of custom GPU operators (a 5x gain) (p. 6). These efforts were then amplified at the hardware layer, where migrating the optimized workload to specialized GPUs provided an additional 10.1x improvement in energy efficiency. This case study effectively demonstrates that substantial sustainability gains are achieved when optimizations are made cohesively across all levels of the technology stack.

\citet{Zhuk2023}'s article significantly contributes to the field by emphasizing the hidden ecological costs and linking AI sustainability with environmental justice and regulatory requirements. The holistic framing invites multidisciplinary collaboration and calls for binding ethical-legal frameworks, renewable energy adoption, and responsible e-waste policies. This foundational work lays the groundwork for future studies to combine qualitative depth with rigorous quantitative lifecycle analyses and policy modeling, fostering AI innovation aligned with ecological and social sustainability goals.

\section{Conclusion}
Conclusion of the work presented in the reviewed research papers.

\section{Future work}
Some suggestions for improvement.

%References
\printbibliography

\end{document}
