\documentclass[a4paper, 12pt]{article}
\usepackage{apacite}
\usepackage[T1]{fontenc}
\usepackage[utf8]{inputenc}
\usepackage{mathptmx}
\usepackage{enumerate}
\usepackage[margin=0.5in]{geometry}


\renewcommand{\baselinestretch}{1.0}

\newcommand\nd{\textsuperscript{nd}\xspace}
\newcommand\rd{\textsuperscript{rd}\xspace}
\newcommand\nth{\textsuperscript{th}\xspace} %\th is taken already

\setlength\parindent{0pt} % set paragraph indent to zero

% fill up your name, ID and paper title here
\author{
IZZMINHAL AKMAL BIN NORHISYAM \quad 242UC240JF \quad contribution1 \\
CHOW YING TONG \quad 242UC244NK \quad contribution2\\
Student Name3 \quad Student ID3 \quad contribution3\\
Student Name4 \quad Student ID4 \quad contribution4\\
}
\title{ REVIEW REPORT  Title  }


\begin{document}
\maketitle


\section{Introduction and Problem Statement}
Write the introduction here. What is the background of the problem? Why is there a problem? What are the previous work/research to attempt the problem? 

Problem statement in a new paragraph, follow by objectives.

Citation example: \cite{Termenchy15} and \cite{Jones19}.

\section{Literature Review}
\subsection{Holistically Analyzing the Carbon Footprint of AI}
\hspace{24pt}The work by \textcite{Wu2022} provides a foundational, holistic analysis of the sources of AI's greenhouse gas emissions, shifting the focus beyond prior research that was often limited to the carbon footprint of training a single, massive model. The authors contend that this narrow, model-centric view has become insufficient in the face of the "super-linear growth" of the entire AI ecosystem, a trend characterized by exponential increases in model complexity, vast volumes of training data, and the ever-expanding hardware infrastructure required to support them. To provide a more accurate assessment, their analysis expands the scope to encompass two critical dimensions: the end-to-end machine learning pipeline (from data acquisition to inference) and the full life cycle of the hardware itself. Critically, the authors introduce a practical framework that distinguishes between \textbf{operational carbon}, defined as the emissions from electricity consumed during use, and \textbf{embodied carbon}, which are the often-hidden emissions generated from the entire manufacturing supply chain.

\section{Research Methodology}
\subsection{Data Collection and Sourcing}
Based on the study conducted by Wu et al. (2022), the data was collected from multiple sources, from internal production systems at Meta, external public reports, and experimental modeling to create a holistic view of AI's environmental footprint. The primary data source consisted of extensive internal infrastructure telemetry from Meta's large-scale AI systems, encompassing operational metrics such as data center power consumption logs, GPU utilization rates, model training job durations, and data ingestion bandwidth statistics. To account for the manufacturing footprint, the authors sourced Life Cycle Inventory (LCI) data from external databases and manufacturer publications, such as Apple's Product Environmental Reports, which served as a necessary proxy for their own hardware components. For the purpose of comparative analysis, the study also incorporated the publicly available carbon footprint results of other prominent models, including GPT-3 and Meena, to contextualize their own findings. Finally, to model the energy consumption of specific edge-computing use cases like federated learning, the authors constructed their estimations using a combination of experimental measurements, including device power profiles from Android and network bandwidth data from public sources.

\subsection{Core Methodology and Algorithms} 


\section{Result}
Result  in a new paragraph.

\section{Discussions}
Discussion in a new paragraph.

\section{Conclusion}
Discussions include but not limited to:
\begin{enumerate}[(a)]
\item Pros and cons of the method/algorithm used,
\item limitation of the method/algorithm,
\item contribution of the research.
\end{enumerate}
Conclusion of the work presented in the reviewed research papers.

\section{Future work}
Some suggestions for improvement.

%References
\bibliographystyle{apacite}
\bibliography{MyBib}{}

\end{document}
